\section{Intro Lecture}

\subsection{Meta information}
22.10.2015

Asaf Nachmias
Schreiber 335
Email: asafnach@pos.tau.ac.il, or asafnach@gmail.com  (For quicker answer)


A list of papers was given at the beginning of the lesson.
Could be also found here:
\url{http://www.math.tau.ac.il/~asafnach/gradseminar2015.html}


\subsection{The first two articles}

$n \geq 1$. $S_n$ permutation.
$T = \{Transpositions\} \subseteq S_n$.
$|T| = \frac{n}{2}$.

Fix $t \geq 1$. Let $T_1, \dots T_t$ be i.i.d $unif(T)$.
Put ${\Pi}_t = T_t \circ \cdots \circ T_1$.


Every permutation is a composition of transpositions.

\begin{thm} (Schramm)
  $t = cn$, $c > 1/2$ const, let $(y_1,y_2,\dots)$ be the ordered cycles sizes
  of ${Pi}_t$, then $(y_1/n,y_2/n,\dots) \overset{d}{\rightarrow} (X_1,X_2,\dots)$
    (Poisson-Dirichlet distribution).
\end{thm}

Analogy: Look at the random graph of $n$ vertices, and we put an edge $(i,j)$ if
there is $T_k$, $1 \leq k \leq t$ which is the transposition $(i,j)$.


\begin{enumerate}
  \item If $c < 1/2$ then all the connected components in the graph are smaller
    than $\log(n)$.
  \item If $c > 1/2$, then there is one linear (Of linear size) connected
    components, and all the rest are of logarithmic size.
\end{enumerate}

We are not sure if all the transpositions inside the giant connected component
will compose together to be one cycle.

\begin{thm}[81, Diaconis-Shashahani]
  \[
    \norm{{\Pi}_t - {\Pi}}_{TV} = \left\{ \begin{array}{ll}
    o_n(1) & c > 1 \\
    1-o_n(1) & c < 1 \end{array}\right.
  \]
\end{thm}

\begin{thm}[Berestycki]
Simpler proof and more general for $k$-cycles.
\end{thm}


\subsection{Articles 3 and 4}

$G = (V,E)$ is inifinitely connected graph.
For every $K \subseteq V$, $|K| < \infty$, $d_E{K} : \{\mbox{Edges with one
endpoint in $K$}\}$.

Cheeger constant:
${\Phi}_E{G} = \inf \{ \frac{|d_E{K}}{|K|} : K \subseteq V Finite, k \neq
\emptyset\}$.

If ${\Phi}_E(G) > 0$ then $G$ is non amenable.

A grid graph is not amenable (The cheager constant will be $0$), however the
${\Phi}_E(\mathcal{T}_d) = d-2$.


For a graph $G$ we define:

\[ P(x,y) = \left\{ \begin{array}{ll}
    1/deg(x) & x \sim y \\
    0 & \mbox{otherwise} \end{array}\right.
\]

$(Pf)(x) = {\sum}_y p(x,y)f(y)$

$\norm{P}_{op} = sup \{ \frac{\norm{Pf}_2}{\norm{f}_2} : f \neq 0\}$

$\norm{P}_{op} \leq 1$


\begin{prop}
$G$ Cayley's graph.
$p^n(x,x) := $ Probability after $n$ steps is at $x$.

$\rho = {\lim}_{n \rightarrow \infty} (p^{2n}(x,x))^{(1/2n)} = \norm{P}_{op}$.
\end{prop}

\begin{thm}
  ${{{\Phi}_E}^2}/2 \leq 1 - \rho \leq {\Phi}_E$

\end{thm}

$\rho(T_d) = \frac{2\sqrt(d-1)}{d}$.

\begin{thm}[Kesten 64']
  $G$ is a cayley graph ($d$-regular), $\rho(G) \geq \rho(T_d) = 2\sqrt{d-1}/d$.
  And $\rho(G) = 2\sqrt{d-1}/d$ iff $G=T_d$.

\end{thm}

Extensions: (Paper 3): Simpler proofs and some more general graphs.


\subsection{The Percolation articles}

Percolation: $G$ infinite connected graph, $p \in [0,1]$.
Delete each edge with probability $1-p$, retain with probability $p$.
$G$ is a cayley graph.

$E$: There exists infinite connected component. (Check that this event is
measurable).

By the Kolmagorov theorem, the probability of this event must be either $0$ or
$1$.

A result by Kesten-Harris: $p_c(\mathcal{Z}^2) = 1/2$.


When is $p_c < 1$? $\mathcal{Z}$.
Conjecture: Every Cayley graph of growth $>1$, $p_c < 1$.
It is known when $G$ has polynomial growth. $|B(0,r)| \leq Cr^d$, and when $G$
is non amenable. Also known when $G$ has exponential growth.
