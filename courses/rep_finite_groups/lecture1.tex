Representation Theory

\part{Intro}

Book to get:

Jean Pierre Serre

Linear representation of finite groups

Amazon.com 45 dollars.
bookzz.org (no money).

Abel prize 2003.


\part{}

Let $k$ be a field $k=\mathbf{C}$. $V$ is a vector space of a finite dimension.
$GL(V)$ is the group of automorphisms of $V$.

Automorphism is:
$a: V \rightarrow V$ such that there is $a^{-1}$.

Let $e_1,\dots,e_n$ a basis of $V$.

$a(e_j) = \sum_{i} a_{ij}e_i$

$a$ is isomorphism if $det(a_{ij}) \neq 0$.

$GL(V)$ matrices $n\times n$ with $det \neq 0$.


$G$ is a finite group. $1 = 1_G$. A rule of multiplication $(s,t) \rightarrow
st$.

\begin{defn}
  A linear representation of $G$ is a homomorphism $\ro : G \rightarrow GL(V)$.
  It is possible to have something in the kernel of the homomorphism.

  This means: For every $s \in G$ we write $\ro(s) \in GL(V)$ such that $\ro(st)
  = \ro(s)\ro(t)$ for every $s,t \in G$.
\end{defn}

\uline{Note:}

So:
$\ro(1_G) = 1_V$.
$\ro(s^{-1}) = \ro(s)^{-1}$.
$1_V := 1_GL(v)$.


$\ro$ representation. We say that $V$ is the space of representation. (Or that
$V$ is a representation).

\uline{Examples:}

\begin{enumerate}
  \item $V = \mathcal{C}^n$. $G$ is some group. $\ro s \rightarrow 1_V$.
    $\ro(st) = 1_V$, $\ro(s) = \ro(t) = 1_V$. $1_V \cdot 1_V = 1_V$.
    This is the trivial representation of dimension $n$.

  \item $G = \{1,-1\}$. $V = \mathcal{C}^n$. $\ro(1) = 1_V$. $\ro(-1) = -(1_V)$.

  \item $G = \{1,-1\}$. $V = \mathcal{C}^2$. 
    \[
      e_1 = \matrix(1 \\
                    0)
                  \]

    \[e_2 = \matrix(0 \\
                    1
                  \]

    $1_G \rightarrow 1_V$.

    \[
      -1 \rightarrow \matrix(0 1 \\
                            1 0) 
  \]

  $(-1)^2 = 1$.


  $G \rightarrow GL_2(\mathcal{C})$
  $G \rightarrow GL(V)$.

  \item $G = \{1,-1\}$
    $V = \mathcal{C}^2$
    $1_G \rightarrow 1_V$.
    $-1 \rightarrow \matrix(-1 0\\
                            0  1)$


\end{enumerate}

$V$ a finite dimensional vector space, of dimension $n$.
$\ro: G \rightarrow GL(V)$.

$e_1,e_2$ base.


$R_{st} = R_s \cdot R_t$ for every $s,t \in G$.
$r_ik(st) = \sum_{j} r_{ij}(s) \cdot r_{jk}(t)$.

Multiplication of matrices?


Let $G$ be a group. Let $(\ro,V)$ and $(\ro',V')$ two different representations
of $G$.

$\ro: G \rightarrow GL(V)$.
$\ro': G \rightarrow GL(V')$.

Are isomorphic if there exists an isomorphism:

$\tau: V \rightarrow V'$ such that $\tau$ changes $\ro$ to $\ro'$.

$\tau \compose \ro(s) = \ro'(s) \compose \tau$

$V$ -- $\ro(s)$ --> $V$
 |
 $\tau$             |
 |                  $\tau$
 V                  V
 $V$ -- $\ro'(s)$ --> $V'$

Commutative.
$\ro'(s) \compose \tau = \tau \compose \ro(s)$ for every $\s in G$.

Exercise: Build $\tau$ for examples 3,4.

0 1     -1 0
1 0      0 1

Those two are similar on $\mathcal{C}, \mathcal{R}$, but not on $\mathcal{Z}$.



