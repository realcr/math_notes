\part{Intro}

22.10.2015

Boolean functions and error correcting codes (Shmuel Safra)
There should be notes for the entire course.

Course requirements:

- A test in the end of the semester.
  - Maybe a smaller test in the middle?

- Homework. Will not get credit. No checker for the homework.
  Probably every two weeks.

- Difficult problem (Star problem) in the middle of the lecture.

Topics that we will learn in this course:

\begin{enumerate}
  \item ECC (Error correccting codes)
  \item Boolean functions  $f: (\{0,1\}}^n \rightarrow \{0,1\)$
  \item Learning (Finding things about Boolean functions quickly).
  \item Cryptography
  \item Hardness of approximation
\end{enumerate}

\uline{Boolean functions:}

Interesting functions: xor, majority, dictatorship (One bit of the input
is the result). Majority is not so sensitive to noise. Majority stablest
theorem.

\begin{defn}
The influence of the variable $x_i$.
$I_i[f] = {\Pr}_x\left[{f(x) \neq f(x^i)}\right]$
The sum of influences is $I[f] = \sum{I_i[f]}$.
\end{defn}

Dicataorship has influence of $1$.

What is the influence value of majority? About $\sqrt(n)$.

\begin{defn}
A monotone function is a function that when changing a variable from $0$ to $1$
does not change its value from $1$ to $0$.
\end{defn}

The idea of functions where only small amount of variables determines the final
value of the function. (Hunta?)

\uline{Learning}

Learning boolean functions.
PAC (Probably approximately correct).

(PAC learning for abelian groups, fourier sparse)

Sampling a few values and building a similar function that has similar
properties?

Some functions can not be approximated using this method (For example, random
functions).

Functions with low total influence might be easier to learn.

Supervised learning. The ability to give challenges to a function. This is not
always what we have in real life. (Unsupervised learning).

\uline{Cryptography}

One way functions. Public key Cryptography: One way functions with trapdoor.

\begin{defn}
One way function: $f: X \rightarrow X$
Given $f(x)$ it is hard to compute $x'$ such that $f(x')=f(x)$
\end{defn}


\uline{Hard Core Predicate}

Sidenote?
The discrete log problem:
Given $g^x \mod p$ : Find $x$.

Given a one way function, we want to be able to find a bit that is hard to
discover. (A stronger requirement of a one way function).

\part{Error Correcting Codes}

We write bits in memory. We want to be able to recover the information in case
some of the bits are changed (Errors).

Important parameters: 
\begin{enumerate}
  \item $m$: The length of the block. We have strings of bits of this size.
  \item $q$: The size of the alphabet. The alphabet set is $[q] =
    \{0,1,\cdots,q-1\}$. In many cases we will work with the binary alphabet:
    $[2]$.
  \item $C \subseteq [q]^m$, the subset of valid code words.
  \item $d$: Distance: $d(C) := \min{x,y \in C}[{\delta}_H (x,y)]$. ${\delta}_H$
    is the hamming distance. We should be able to fix at most $d/2$ errors.
    (TODO: A picture $[q]^m$ with balls around code words of radius $d/2$).

  \item The rate of the code $C$: $r[C] = \log_{q}{|C|} / m$.
\end{enumerate}


\section{Linear Error correcting codes}

Assume that we have a finite field $Q$, where $|Q| = q$.
Examples for finite fields: $\mathcal{Z}_p$, where $p$ is prime.
We have such finite fields only of sizes of the form $p^r$, where $p$ is some
prime. All the finite fields of the same size are isomorphic. (Galois theory).

$Q^m$ is a vector space.
